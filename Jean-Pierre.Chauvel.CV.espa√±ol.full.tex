%% start of file `Jean-Pierre.Chauvel.CV.español.full.tex`
%% Copyright 2016 Jean-Pierre Chauvel <jean.pierre@chauvel.org>

\documentclass[11pt,a4paper,english]{moderncv}
% moderncv theme
\moderncvtheme[grey]{casual}
% document language
\usepackage[english]{babel}
% character encoding
\usepackage[utf8]{inputenc}
% adjust page margins
\usepackage[scale=0.8]{geometry}
% in preamble:
\SetExpansion[stretch = 70, shrink = 70,] { encoding = {T2A} } { }
\DeclareMicrotypeSet{t2atext}{encoding=T2A}
\UseMicrotypeSet{t2atext}

%personal data
\name{Jean-Pierre}{Chauvel}
\address{Lima, Perú}{}
\email{jean.pierre@chauvel.org}
\homepage{www.chauvel.org}
%\social[twitter]{hellhoundorf}
\social[whatsapp]{+51989804478}
\social[linkedin]{jpchauvel}
\social[github]{jpchauvel}
\quote{Con casi dos décadas de experiencia en la creación de soluciones empresariales, me destaco como un defensor de las metodologías ágiles con una sólida experiencia en backend de Python. Mi afinidad por Vim ejemplifica mi fusión de rigor clásico con agilidad contemporánea.}

\begin{document}
\maketitle

\section{Experiencia Reciente}

\cventry{2017--Presente}{Ingeniero de Software Senior en Python}{Encora}{Lima--Perú}{}
{
    \url{https://encora.com}
\newline{}
\begin{itemize}
    \item \textbf{AndGo de Goodyear}
        \newline{}
        \url{https://www.andgonow.com}
        \begin{itemize}
            \item Lideré la creación de especificaciones técnicas completas y diagramas para concretar los requisitos del producto, facilitando la comunicación fluida entre desarrolladores y partes interesadas.
            \item Implementé Test-Driven Development (TDD) con diligencia, manteniendo la cobertura de código por encima del 90\% para pruebas unitarias, de integración y de extremo a extremo, asegurando entregas de software robustas y confiables.
            \item Desempeñé un papel crucial en la descomposición de un servicio monolítico en una suite de microservicios enfocados, cuidando meticulosamente la reestructuración del esquema de la base de datos para escalabilidad y resiliencia.
            \item Lideré el desarrollo de una característica crítica para la automatización de eventos de calendario, mejorando la eficiencia del proceso mediante una planificación anticipada meticulosa.
        \end{itemize}
        \textbf{Herramientas:} Python, asyncio, Flask, SQLAlchemy, PostgreSQL, Mock, Git, Docker, Microservicios
    \item \textbf{MPACT de Tribal Planet}
        \newline{}
        \url{https://www.tribalplanet.com}
        \begin{itemize}
            \item Orquesté el plano arquitectónico y lideré el desarrollo del backend, manteniendo una infraestructura de aplicación de vanguardia.
            \item Elevé la cobertura de pruebas con un enfoque de TDD para un entorno de producción impermeable y resistente a errores.
            \item Innové un robusto pipeline de despliegue continuo utilizando CircleCI y AWS EKS para agilizar la entrega de software.
            \item Mejoré el rendimiento descargando tareas computacionalmente intensivas a trabajos de Celery, optimizando operaciones asíncronas.
            \item Desarrollé un sistema automatizado de moderación de contenido utilizando AWS Rekognition y Azure Cognitive Services que mantuvo la calidad y el cumplimiento del contenido.
        \end{itemize}
        \textbf{Herramientas:} Python, asyncio, Flask, SQLAlchemy, MySQL, MongoDB, Mock, Git, Docker
    \item \textbf{Banca móvil de Scotiabank}
        \begin{itemize}
            \item Contribuí al desarrollo de la aplicación iOS usando Objective-C, asegurando la adherencia a las mejores prácticas de la industria y una experiencia de usuario consistente.
            \item Mantuve los principios de TDD para garantizar una cobertura completa de pruebas unitarias, fomentando la fiabilidad en las liberaciones de la aplicación.
            \item Utilicé el modelo de ramificación GitFlow para mantener un ritmo de versionado de software estable y metódico.
        \end{itemize}
        \textbf{Herramientas:} Objective-C, Swift 4.1, Carthage, Git
\end{itemize}
}

\subsection{}

\cventry{2017}{Ingeniero de Software iOS}{Dipoo SAC}{Lima--Perú}{}
{
\begin{itemize}
    \item \textbf{Bederr}
        \begin{itemize}
            \item Estuve activo en la mejora y expansión de la aplicación iOS mediante el uso de Objective-C, asegurando meticulosamente que cada módulo se alinee con los estándares de la industria y las mejores prácticas vigentes, proporcionando así una experiencia de usuario uniforme y sin interrupciones en diversas plataformas de dispositivos.
            \item Me Adherí diligentemente a las metodologías de Desarrollo Guiado por Pruebas (TDD) para asegurar la implementación de pruebas unitarias robustas, lo cual contribuye significativamente a la estabilidad y fiabilidad de la aplicación, minimizando así las interrupciones en la producción y mejorando la satisfacción del usuario con cada lanzamiento de software.
            \item Implementado la estrategia de ramificación GitFlow para gestionar sistemáticamente los procesos de desarrollo y lanzamiento, asegurando un enfoque disciplinado y consistente en la versionado del software, lo que facilita transiciones y actualizaciones más suaves en el ciclo de vida de la aplicación, mejorando la colaboración del equipo y la eficiencia en el manejo de los plazos del proyecto.
        \end{itemize}
        \textbf{Herramientas:} Objective-C (iOS), OCMockito, Core Location, Google Maps, Circle CI, Hock- eyApp, CocoaPods
\end{itemize}
}

\subsection{}

\cventry{2016-2017}{Ingeniero de Software iOS}{Freelance}{Lima--Perú}{}
{
\begin{itemize}
    \item \textbf{Ensayos Impopulares de Aldo Mariátegui}
        \begin{itemize}
            \item Contribuí al desarrollo de la aplicación para iOS utilizando Objective-C, asegurando la adherencia a las mejores prácticas de la industria y una experiencia de usuario consistente.
            \item Mantuve los principios de TDD para garantizar una cobertura completa de las pruebas unitarias, fomentando la fiabilidad en las versiones de la aplicación.
            \item Utilicé el modelo de ramificación GitFlow para mantener un ritmo de versionado de software estable y metódico.
        \end{itemize}
        \textbf{Herramientas:} Objective-C (iOS), AFNetworking, OCUnit (SenTestingKit), OCHamcrest, Python, Django, Django RESTFramework, nose, Mock, Gunicorn, Supervisord, XML, Travis-CI, Fabric
\end{itemize}
}

\subsection{}

\cventry{2016}{Ingeniero de Software Senior en Python}{Freelance}{Lima--Perú}{}
{
    \begin{itemize}
        \item \textbf{Copa Master}
            \begin{itemize}
                \item Fui pionero en el ciclo de vida completo del proyecto desde la arquitectura hasta el mantenimiento del backend, enfatizando en una estrategia de desarrollo ágil y adaptable.
                \item Adopté un enfoque centrado en TDD asegurando el control de calidad a través de pruebas exhaustivas, lo que llevó a un proceso de despliegue sin contratiempos.
                \item Diseñé y ajustó un pipeline de despliegue continuo, transitando fluidamente de CircleCI a AWS EKS, maximizando la eficiencia y fiabilidad del despliegue.
            \end{itemize}
            \textbf{Herramientas:} Python, Django, PostgreSQL, JQuery, nose, Mock, Gunicorn, Supervisord, HTML5, CSS3, Git, Docker
    \end{itemize}
}

\subsection{}

\cventry{2014}{Ingeniero de Software Senior en Python}{PASTPRESENTFUTURE}{Lima--Perú}{}
{
    \url{https://pastpresentfuture.com}
\newline{}
\begin{itemize}
    \item \textbf{Doccupations}
        \newline{}
        WIP (\url{http://doccupations.com})
        \begin{itemize}
            \item Fomenté una cultura de desarrollo ágil mediante la capacitación a clientes y colegas, mejorando la dinámica de equipo y los resultados del proyecto.
            \item Lideré la concepción del proyecto, diseñando una arquitectura de backend resistente y escalable que resultó en una impresionante cobertura de código del 90\% mediante pruebas rigurosas.
            \item Desarrollé un robusto pipeline de despliegue continuo con CircleCI, optimizando el ciclo de vida de lanzamiento del software.
            \item Diseñé e implementó un sistema de entrega de correos electrónicos asíncrono utilizando Celery con RabbitMQ, mejorando dramáticamente la eficiencia y la capacidad de respuesta de las funciones de comunicación.
            \item Mejoré las capacidades de búsqueda de texto mediante la integración de Apache Solr, proporcionando una experiencia de usuario superior con resultados de búsqueda rápidos y precisos.
        \end{itemize}
        \textbf{Herramientas:} Python, Django, Celery, RabbitMQ, PostgreSQL, Apache Solr, JQuery, nose, Mock, Gunicorn, Supervisord, Stripe, HTML5, CSS3, Git, CircleCI, Fabric
\end{itemize}
}

\subsection{}

\cventry{2013--2124}{Ingeniero Senior de Software en Python}{Bit Zeppelin}{Lima--Peru}{}
{
\begin{itemize}
    \item \textbf{Un sistema ERP sin nombre}
        \begin{itemize}
            \item Participé en el complejo proceso de arquitectura y desarrollo de una sofisticada interfaz web para el sistema ERP de código abierto y anteriormente solo de escritorio conocido como Tryton, lo que requirió la ingeniería inversa de la API debido a la ausencia de documentación técnica preexistente relacionada con las interfaces de software cliente.
            \item Implementé una estrategia de pruebas exhaustiva, que incluyó tanto pruebas unitarias como pruebas de extremo a extremo, para asegurar una cobertura robusta y mantener la integridad y fiabilidad de la base de código a lo largo del ciclo de desarrollo.
        \end{itemize}
        \textbf{Herramientas:} Python, Django, Tryton, MySQL, Scrapy, jQuery, HTML5, CSS3, Git, CircleCI, Fabric
    \item \textbf{Una aplicación web de crédito hipotecario}
        \begin{itemize}
            \item Asumí la total responsabilidad del mantenimiento del backend de una aplicación web crítica de crédito hipotecario, centrando mis esfuerzos en mejorar el rendimiento y la estabilidad del sistema a través de meticulosas revisiones y optimizaciones del código.
            \item Dedicado a mantener una alta calidad del código, apliqué rigurosamente metodologías de pruebas unitarias a todas las modificaciones para asegurar que cada cambio cumpliera con los estándares predefinidos y no introdujera regresiones ni nuevos errores.
        \end{itemize}
        \textbf{Herramientas:} Python, Tornado, MongoDB, jQuery, HTML5, CSS3
    \item \textbf{Bolsa de trabajo de la UPC}
        \begin{itemize}
            \item Encargado del mantenimiento continuo de los sistemas de backend para la plataforma de banca móvil de Scotiabank, enfocándome en optimizar el manejo de datos y el procesamiento de transacciones para mejorar la experiencia del usuario y la fiabilidad del sistema.
            \item Comprometido con asegurar la estabilidad y fiabilidad de la plataforma, empleé pruebas unitarias en todas las nuevas implementaciones de código para validar la funcionalidad y prevenir interrupciones en el servicio, adhiriéndome a estrictos protocolos de aseguramiento de calidad.
        \end{itemize}
        \textbf{Herramientas:} Python, Django, PostgreSQL, Gunicorn, Supervisord, HTML5, CSS3
\end{itemize}
}

\subsection{}

\cventry{2012--2123}{Ingeniero de Software Senior en iOS y Python}{Taller Technologies}{Lima--Peru}{}
{
\begin{itemize}
    \item \textbf{TList}
        \begin{itemize}
            \item Como arquitecto de software y principal ingeniero del proyecto, lideré el desarrollo del frontend de iOS, diseñando meticulosamente una arquitectura robusta que se integraba a la perfección con los servicios de backend, asegurando una experiencia de usuario fluida y receptiva mientras se adhería a las mejores prácticas en calidad de código y mantenibilidad.
            \item Las funcionalidades principales que desarrollé para la aplicación incluyeron una interfaz de chat avanzada que permitía a los usuarios comunicarse mediante texto y emojis, mejorando el compromiso del usuario. Además, implementé un sistema de perfil de usuario dinámico donde los individuos podían publicar estados visibles para su red, y una interfaz de búsqueda sofisticada que permitía a los usuarios localizar compañeros dentro de sus propias instituciones educativas y afiliadas, fomentando así un ambiente comunitario conectado.
            \item Durante un período en el que Xcode carecía de capacidades de despliegue continuo integradas, tomé la iniciativa de construir un completo pipeline de CI desde cero. Utilizando Jenkins junto con técnicas avanzadas de scripting en bash, establecí un flujo de trabajo de despliegue robusto que se interfazaba con el servicio entonces independiente TestFlight, antes de su adquisición por Apple, agilizando así el proceso de desarrollo y mejorando la productividad.
        \end{itemize}
        \textbf{Herramientas:} Objective-C, AFNetworking, XMPPFramework, OCUnit, OCHamcrest, Frank, TestFlight
    \item \textbf{Context Aware Computing}
        \begin{itemize}
            \item Como un contribuyente clave al proyecto, mi rol se centró principalmente en aprovechar mi experiencia en Python para arquitecturar y refinar funcionalidades clave, asegurando prácticas de desarrollo robustas mientras también facilitaba la transferencia de conocimientos como formador de Python para elevar la acuidad de programación del equipo.
            \item Me encargaron el desarrollo de componentes críticos para la misión, específicamente utilizando Interfaces de Zope para mejorar la modularidad y reutilización a través de la arquitectura de software, asegurando así que la escalabilidad y mantenibilidad del sistema cumplieran con los estrictos requisitos del alcance de nuestro proyecto.
            \item Mis responsabilidades también incluyeron la realización de sesiones de capacitación exhaustivas centradas en metodologías de pruebas unitarias, de integración y de extremo a extremo utilizando bibliotecas de pruebas de Python avanzadas, elevando significativamente la competencia del equipo en la entrega de código de alta calidad y libre de errores en alineación con las mejores prácticas de la industria.
        \end{itemize}
        \textbf{Herramientas:} Python, Zope, Atlassian Bamboo
    \item \textbf{Staff Allocation Tool}
        \begin{itemize}
            \item Como Ingeniero de Software Junior, participé activamente en el proyecto, aprovechando mi conocimiento fundamental en Java para contribuir efectivamente a las tareas de desarrollo, a pesar de mi estado inicial como novato en el lenguaje de programación, lo cual proporcionó una plataforma robusta para mi crecimiento profesional y mejora de habilidades.
            \item Empleé extensamente Groovy como un lenguaje "pegamento" integral para facilitar procesos de despliegue continuo sin interrupciones, mientras utilizaba Maven como una herramienta sofisticada de gestión de dependencias para asegurar construcciones de proyectos consistentes y agilizar la integración de varios componentes de software.
            \item Asumí un papel clave en aseguramiento de la calidad al orquestar la fase de pruebas exploratorias, examinando meticulosamente e identificando posibles discrepancias y áreas de mejora dentro del software para mantener y mejorar la fiabilidad del producto y la satisfacción del usuario.
        \end{itemize}
        \textbf{Herramientas:} Java, Spring, PrimeFaces, Maven, Groovy, Selenium, XML, Atlassian Bamboo
\end{itemize}
}

\subsection{}

\cventry{2010--2012}{Ingeniero de Software Senior en iOS y Python/Gerente de Proyecto/Socio de la Empresa}{Bit Zeppelin}{Lima--Peru}{}
{
\begin{itemize}
    \item \textbf{Aplicación de Teleticket}
        \begin{itemize}
            \item Implementé la biblioteca ASIHTTP para descargas asíncronas en un hilo secundario, mejorando el rendimiento de la aplicación y la experiencia del usuario durante transferencias intensivas de datos.
            \item Desarrollé una característica utilizando algoritmos personalizados para la descarga autónoma y sincronización de datos de eventos, asegurando una interfaz de usuario actualizada y precisa, mejorando el compromiso del usuario.
            \item Ingenié funcionalidad de mapeo con MapKit, colocando pines para los lugares de eventos en un mapa digital, proporcionando una visualización geográfica intuitiva, mejorando la navegación y planificación del usuario en la aplicación.
            \item Dirigí la recopilación de requisitos y transformé especificaciones complejas en historias de usuario accionables, agilizando el desarrollo y alineándolo con las expectativas de los interesados y las necesidades del usuario.
            \item Supervisé un equipo de cinco desarrolladores, liderando reuniones diarias y ceremonias de scrum, fomentando un ambiente colaborativo, acelerando los plazos y mejorando la productividad a través de la comunicación y la retroalimentación.
            \item Comuniqué regularmente el progreso y las actualizaciones al cliente quincenalmente y después de cada sprint, asegurando transparencia y manteniendo relaciones sólidas al mantenerlos informados sobre los hitos y decisiones estratégicas.
        \end{itemize}
        \textbf{Herramientas:} Objective-C, Three20, ASIHTTP
    \item \textbf{tootalk}
        \begin{itemize}
            \item Desarrollé e implementé el códec Speex en C para iOS y Android, mejorando la compresión y descompresión de audio para comunicaciones móviles.
            \item Lideré un rebranding de Linphone para tootalk, alineando su diseño y funcionalidad con la identidad corporativa, mejorando el compromiso de los usuarios y la consistencia de la marca.
            \item Me centré en extraer y documentar los requisitos del software, traduciéndolos en historias de usuario para el backlog del sprint, optimizando la gestión de proyectos y el desarrollo ágil.
            \item Como Arquitecto de Software y Gerente de Proyecto, tomé decisiones clave sobre los protocolos de comunicación de códecs y realicé pruebas de carga para asegurar alta calidad de audio y rendimiento.
            \item Gestioné y mentoré a cuatro Desarrolladores Junior en la integración de la marca corporativa en la aplicación, asegurando la alineación con la estrategia de marca y los estándares de identidad corporativa.
        \end{itemize}
        \textbf{Herramientas:} Objective-C, Three20, Linphone, C
    \item \textbf{mitiempo.pe}
        \begin{itemize}
            \item Realicé mi primer proyecto en iOS, aprendiendo rápidamente Objective-C y Cocoa Touch. Después de un mes de entrenamiento intensivo, comencé a desarrollar aplicaciones iOS complejas utilizando los frameworks de Apple.
            \item Inicialmente utilicé una gestión de proyectos de estilo cascada debido al limitado conocimiento en metodologías ágiles. Este enfoque condujo a estimaciones de proyectos especulativas sin retroalimentación iterativa ni mejoras.
            \item Las pruebas de la aplicación se realizaron manualmente, involucrando simulaciones detalladas de interacciones de usuarios para encontrar errores y evaluar la usabilidad, sin utilizar frameworks de pruebas automatizadas.
            \item El desarrollo de backend subcontratado requirió una coordinación y comunicación estrictas para integrar servicios de terceros de manera fluida con las operaciones de frontend, asegurando el cumplimiento de las especificaciones y los plazos del proyecto.
        \end{itemize}
        \textbf{Herramientas:} Objective-C, Facebook’s Three20
\end{itemize}
}

\subsection{}

\cventry{2010}{Ingeniero de Software Senior en Python y C\#}{ICTEC}{Lima--Peru}{}
{
\begin{itemize}
    \item \textbf{Neoauto}
        \begin{itemize}
            \item Participé en una sesión de mantenimiento integral destinada a mejorar la eficiencia operativa del sistema, con un enfoque específico en la integración del módulo avanzado de publicidad. Esta integración no solo optimiza los algoritmos de colocación de anuncios, sino que también aprovecha técnicas de segmentación de vanguardia para mejorar el compromiso del usuario y las capacidades de generación de ingresos.
        \end{itemize}
        \textbf{Herramientas:}  Python, Django, MySQL, jQuery, HTML, CSS, Subversion
    \item \textbf{VHF Radio Device Driver for Orion}
        \begin{itemize}
            \item Aprovechando las capacidades fundamentales de una interfaz de software existente que facilitaba la comunicación con el sistema de radio, diseñé un prototipo avanzado para el controlador de Orion utilizando Python. Este prototipo incorporó la replicación estratégica de la carga útil del software mencionado, permitiendo así la interoperabilidad sin problemas entre sistemas de radio diversos a través del modelo basado en Python, demostrando así el potencial para la integración escalable en tecnologías de comunicación.
            \item Al pasar de la fase de prototipo a la producción, desarrollé meticulosamente la versión definitiva del controlador de Orion usando C\#, enfocándome en la robustez y la eficiencia. Este proceso de desarrollo no solo involucró la codificación, sino también la integración meticulosa y el empaquetado del controlador en el producto final, asegurando que el software cumpliera con todos los requisitos operativos y estuviera listo para su implementación en entornos complejos, mejorando así la infraestructura de comunicación.
            \item En una demostración crítica ante el Comité de Aseguramiento de Calidad del Ejército Peruano, presenté el controlador de Orion completamente desarrollado, articulando sus características, beneficios operativos y cumplimiento con los estrictos estándares militares. Esta presentación, detallada y exhaustiva, culminó en la obtención de la aprobación final del comité, afirmando la preparación del controlador para el despliegue en campo y su alineación con los objetivos de comunicación estratégicos del ejército.
        \end{itemize}
        \textbf{Herramientas:} IronPython, .NET
\end{itemize}
}

\subsection{}

\cventry{2006--2009}{Ingeniero de Software Python de Nivel Mid-Senior}{Aureal}{Lima--Peru}{}
{
\begin{itemize}
    \item \textbf{Neoauto}
        \begin{itemize}
            \item Como líder técnico del proyecto, lideré el proceso de desarrollo, supervisando a un equipo de ingenieros capacitados y asegurando que todos los aspectos técnicos se alinearan con los objetivos estratégicos, facilitando así una ejecución del proyecto eficiente y ordenada.
            \item A pesar de que los componentes de frontend fueron desarrollados externamente por una firma uruguaya especializada, nuestro equipo emprendió una refactorización significativa de las bases de código HTML y CSS para asegurar una integración sin problemas y un rendimiento óptimo dentro del marco de Django, mejorando tanto la mantenibilidad como la escalabilidad de las aplicaciones web.
            \item Participé activamente y promoví las metodologías de Programación Extrema (XP), específicamente el Desarrollo Guiado por Pruebas (TDD) y la programación en pareja, para fomentar un ambiente colaborativo y mejorar la calidad del código, lo que contribuyó significativamente a reducir errores y acelerar el ciclo de desarrollo.
            \item Fui responsable de compilar un conjunto exhaustivo de documentación técnica utilizando Sphinx, que incluía detallados docstrings de Python que sirvieron como una herramienta de referencia crucial para las fases de desarrollo actuales y futuras, asegurando la consistencia y comprensión en todo el equipo técnico.
            \item Desarrollé simultáneamente las plataformas Aptitus y Neoauto, lo que requirió la creación de bibliotecas centrales robustas que se utilizaron en ambos proyectos para agilizar los procesos de desarrollo, reducir la redundancia y mantener un alto estándar de eficiencia y reutilización del código.
        \end{itemize}
        \textbf{Herramientas:} Python 2.5, Django 1.x, OpenID, Memcached, South, MySQL, CSS, HTML, Vanilla Javascript, Subversio
    \item \textbf{Aptitus}
        \begin{itemize}
            \item Como líder técnico del proyecto, encabecé el equipo de desarrollo, guiando estratégicamente las decisiones arquitectónicas y asegurando la integración de tecnologías de vanguardia para cumplir con los objetivos del proyecto, mientras fomentaba un ambiente propicio para la resolución innovadora de problemas y el desarrollo de software de alta calidad.
            \item Participé activamente en la implementación de las prácticas de Programación Extrema (XP), enfocándome específicamente en el Desarrollo Guiado por Pruebas (TDD) y la Programación en Parejas, para mejorar la calidad y fiabilidad del código a través de protocolos de pruebas rigurosos y sesiones de codificación colaborativas, reduciendo significativamente las tasas de errores y acelerando el proceso de desarrollo de características.
            \item Fui responsable de compilar toda la documentación técnica utilizando Sphinx, documentando meticulosamente los docstrings de Python para crear un repositorio de documentación comprensivo y navegable que apoya tanto las necesidades actuales del proyecto como los esfuerzos de mantenimiento futuro, proporcionando explicaciones claras y detalladas de las funcionalidades del código y las APIs.
            \item Desarrollé simultáneamente las plataformas Aptitus y Neoauto, lo que requirió la creación de bibliotecas centrales robustas que se utilizaron en ambos proyectos para agilizar los procesos de desarrollo, reducir la redundancia y mantener un alto estándar de eficiencia y reusabilidad del código.
        \end{itemize}
        \textbf{Herramientas:} Python 2.5, Django 1.x, OpenID, Memcached, South, MySQL, CSS, HTML, Vanilla Javascript, Subversion
    \item \textbf{Americatel's ERP}
        \begin{itemize}
            \item Estuve activo en el mantenimiento sistemático y la mejora de las funcionalidades del software, centrado en la integración y despliegue de nuevas características diseñadas para optimizar el rendimiento del sistema y la interacción del usuario, asegurando la alineación con los objetivos estratégicos del proyecto y los requisitos operativos.
            \item Empleé un sofisticado enfoque de gestión de proyectos de estilo cascada, donde las fases de pruebas exhaustivas fueron estratégicamente programadas para seguir a la finalización de todas las actualizaciones de desarrollo, facilitando así un proceso de validación ágil y eficiente para mantener los más altos estándares de integridad y fiabilidad del software.
        \end{itemize}
        \textbf{Herramientas:} Python 2.4, Django 0.95, MySQL, CSS, HTML, Vanilla Javascript, Subversion
\end{itemize}
}

\subsection{}

\section{Educación}
\cventry{1999--2002}
    {Título de Pregrado}
    {Instituto Superior Tecnológico de Ciencias de la Información}{Lima--Perú}
    {}{Antiguo Instituto de Ciencias de la Información de la Universidad Nacional de Ingeniería}

\subsection{}

\cventry{1996--1998}
    {Diploma}
    {Colegio de la Inmaculada}{Lima--Perú}
    {}{}

\subsection{}

\section{Referencias}
\cvlistitem
{
    \textbf{Robert Werner}
    \newline{}
    Ingeniero de Software Senior en Goodyear Tire \& Rubber
    \newline{}
    \url{https://www.linkedin.com/in/robertmdwerner}
}
\cvlistitem
{
    \textbf{Sandra Quiroz}
    \newline{}
    Jefa de Desarrollo en Encora Lima
    \newline{}
    \url{https://www.linkedin.com/in/sandradiazquiroz}
}
\cvlistitem
{
    \textbf{Breno Colom}
    \newline{}
    Ingeniero de Software en Scrapinghub
    \newline{}
    \url{https://linkedin.com/in/brenocolom}
}
\end{document}
