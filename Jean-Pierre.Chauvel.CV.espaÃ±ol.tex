%% start of file `Jean-Pierre.Chauvel.CV.español.tex`
%% Copyright 2016 Jean-Pierre Chauvel <jean.pierre@chauvel.org>

\documentclass[11pt,a4paper,english]{moderncv}
% moderncv theme
\moderncvtheme[grey]{casual}
% document language
\usepackage[english]{babel}
% character encoding
\usepackage[utf8]{inputenc}
% adjust page margins
\usepackage[scale=0.8]{geometry}
% in preamble:
\SetExpansion[stretch = 70, shrink = 70,] { encoding = {T2A} } { }
\DeclareMicrotypeSet{t2atext}{encoding=T2A}
\UseMicrotypeSet{t2atext}

%personal data
\name{Jean-Pierre}{Chauvel}
\address{Lima, Perú}{}
\email{jean.pierre@chauvel.org}
\homepage{www.chauvel.org}
%\social[twitter]{hellhoundorf}
\social[whatsapp]{+51989804478}
\social[linkedin]{jpchauvel}
\social[github]{jpchauvel}
\quote{Con casi dos décadas de experiencia en la creación de soluciones empresariales, me destaco como un defensor de las metodologías ágiles con una sólida experiencia en backend de Python. Mi afinidad por Vim ejemplifica mi fusión de rigor clásico con agilidad contemporánea.}

\begin{document}
\maketitle

\section{Experiencia Reciente}

\cventry{2024--Presente}{Ingeniero de Software Senior en Python}{Globant}{Lima--Peru}{}
{
\url{https://globant.com}
\newline{}
\begin{itemize}
    \item \textbf{Autodesk}
        \newline{}
        \url{https://www.autodesk.com}
        \begin{itemize}
            \item Encabecé la planificación, el desarrollo y la demostración de un PoC (Prueba de Concepto) de gran impacto utilizando Splunk para un análisis integral de registros, destacando la ingesta de datos en tiempo real rentable, la correlación y la obtención de información procesable para impulsar la toma de decisiones organizacionales y la eficiencia operativa.
            \item Brindé un apoyo y liderazgo fundamentales en la transición de Kibana a Splunk Federated Search, reduciendo meticulosamente la sobrecarga de ingesta de datos y logrando ahorros de costos medibles, al tiempo que optimicé los flujos de trabajo de análisis y habilité capacidades escalables de análisis de registros centralizados.
            \item Concibí, diseñé y ejecuté un robusto PoC que ilustra el proceso de realizar migraciones de esquemas en bases de datos Amazon RDS a través de Cloud OS v3, aprovechando una pila de infraestructura basada en ECS (Python, Flask, SQLAlchemy, Alembic, Docker) desplegada sin inconvenientes en múltiples regiones, demostrando eficazmente la resiliencia y mantenibilidad de la solución en condiciones similares a las de producción.
            \item Demostré competencia técnica al diseñar e implementar un PoC para orquestar migraciones de esquemas en bases de datos RDS a través de Cloud OS v3 en una infraestructura Serverless (Python, Flask, SQLAlchemy, Alembic), desplegándolo en múltiples regiones para promover una escalabilidad rápida, un mínimo costo operativo y un mayor mantenimiento de las migraciones de bases de datos.
        \end{itemize}
        \textbf{Tools:} Splunk, Kibana, Python, Flask, SQLAlchemy, Alembic, MySQL, Mock, Git, Docker, Docker, Serverless, ECS, Amazon RDS
\end{itemize}
}

\cventry{2017--2023}{Ingeniero de Software Senior en Python}{Encora}{Lima--Perú}{}
{
    \url{https://encora.com}
\newline{}
\begin{itemize}
    \item \textbf{AndGo de Goodyear}
        \newline{}
        \url{https://www.andgonow.com}
        \begin{itemize}
            \item Lideré la creación de especificaciones técnicas completas y diagramas para concretar los requisitos del producto, facilitando la comunicación fluida entre desarrolladores y partes interesadas.
            \item Implementé Test-Driven Development (TDD) con diligencia, manteniendo la cobertura de código por encima del 90\% para pruebas unitarias, de integración y de extremo a extremo, asegurando entregas de software robustas y confiables.
            \item Desempeñé un papel crucial en la descomposición de un servicio monolítico en una suite de microservicios enfocados, cuidando meticulosamente la reestructuración del esquema de la base de datos para escalabilidad y resiliencia.
            \item Lideré el desarrollo de una característica crítica para la automatización de eventos de calendario, mejorando la eficiencia del proceso mediante una planificación anticipada meticulosa.
        \end{itemize}
        \textbf{Herramientas:} Python, asyncio, Flask, SQLAlchemy, PostgreSQL, Mock, Git, Docker, Microservicios
    \item \textbf{MPACT de Tribal Planet}
        \newline{}
        \url{https://www.tribalplanet.com}
        \begin{itemize}
            \item Orquesté el plano arquitectónico y lideré el desarrollo del backend, manteniendo una infraestructura de aplicación de vanguardia.
            \item Elevé la cobertura de pruebas con un enfoque de TDD para un entorno de producción impermeable y resistente a errores.
            \item Innové un robusto pipeline de despliegue continuo utilizando CircleCI y AWS EKS para agilizar la entrega de software.
            \item Mejoré el rendimiento descargando tareas computacionalmente intensivas a trabajos de Celery, optimizando operaciones asíncronas.
            \item Desarrollé un sistema automatizado de moderación de contenido utilizando AWS Rekognition y Azure Cognitive Services que mantuvo la calidad y el cumplimiento del contenido.
        \end{itemize}
        \textbf{Herramientas:} Python, asyncio, Flask, SQLAlchemy, MySQL, MongoDB, Mock, Git, Docker
    \item \textbf{Banca móvil de Scotiabank}
        \begin{itemize}
            \item Contribuí al desarrollo de la aplicación iOS usando Objective-C, asegurando la adherencia a las mejores prácticas de la industria y una experiencia de usuario consistente.
            \item Mantuve los principios de TDD para garantizar una cobertura completa de pruebas unitarias, fomentando la fiabilidad en las liberaciones de la aplicación.
            \item Utilicé el modelo de ramificación GitFlow para mantener un ritmo de versionado de software estable y metódico.
        \end{itemize}
        \textbf{Herramientas:} Objective-C, Swift 4.1, Carthage, Git
\end{itemize}
}

\subsection{}

\cventry{2016}{Ingeniero de Software Senior en Python}{Freelance}{Lima--Perú}{}
{
    \begin{itemize}
        \item \textbf{Copa Master}
            \begin{itemize}
                \item Fui pionero en el ciclo de vida completo del proyecto desde la arquitectura hasta el mantenimiento del backend, enfatizando en una estrategia de desarrollo ágil y adaptable.
                \item Adopté un enfoque centrado en TDD asegurando el control de calidad a través de pruebas exhaustivas, lo que llevó a un proceso de despliegue sin contratiempos.
                \item Diseñé y ajustó un pipeline de despliegue continuo, transitando fluidamente de CircleCI a AWS EKS, maximizando la eficiencia y fiabilidad del despliegue.
            \end{itemize}
            \textbf{Herramientas:} Python, Django, PostgreSQL, JQuery, nose, Mock, Gunicorn, Supervisord, HTML5, CSS3, Git, Docker
    \end{itemize}
}

\subsection{}

\cventry{2014}{Ingeniero de Software Senior en Python}{PASTPRESENTFUTURE}{Lima--Perú}{}
{
    \url{https://pastpresentfuture.com}
\newline{}
\begin{itemize}
    \item \textbf{Doccupations}
        \newline{}
        WIP (\url{http://doccupations.com})
        \begin{itemize}
            \item Fomenté una cultura de desarrollo ágil mediante la capacitación a clientes y colegas, mejorando la dinámica de equipo y los resultados del proyecto.
            \item Lideré la concepción del proyecto, diseñando una arquitectura de backend resistente y escalable que resultó en una impresionante cobertura de código del 90\% mediante pruebas rigurosas.
            \item Desarrollé un robusto pipeline de despliegue continuo con CircleCI, optimizando el ciclo de vida de lanzamiento del software.
            \item Diseñé e implementó un sistema de entrega de correos electrónicos asíncrono utilizando Celery con RabbitMQ, mejorando dramáticamente la eficiencia y la capacidad de respuesta de las funciones de comunicación.
            \item Mejoré las capacidades de búsqueda de texto mediante la integración de Apache Solr, proporcionando una experiencia de usuario superior con resultados de búsqueda rápidos y precisos.
        \end{itemize}
        \textbf{Herramientas:} Python, Django, Celery, RabbitMQ, PostgreSQL, Apache Solr, JQuery, nose, Mock, Gunicorn, Supervisord, Stripe, HTML5, CSS3, Git, CircleCI, Fabric
\end{itemize}
}

\subsection{}

\section{Educación}
\cventry{1999--2002}
    {Título de Pregrado}
    {Instituto Superior Tecnológico de Ciencias de la Información}{Lima--Perú}
    {}{Antiguo Instituto de Ciencias de la Información de la Universidad Nacional de Ingeniería}

\subsection{}

\cventry{1996--1998}
    {Diploma}
    {Colegio de la Inmaculada}{Lima--Perú}
    {}{}

\subsection{}
\newpage

\section{Referencias}
\cvlistitem
{
    \textbf{Robert Werner}
    \newline{}
    Ingeniero de Software Senior en Goodyear Tire \& Rubber
    \newline{}
    \url{https://www.linkedin.com/in/robertmdwerner}
}
\cvlistitem
{
    \textbf{Sandra Quiroz}
    \newline{}
    Jefa de Desarrollo en Encora Lima
    \newline{}
    \url{https://www.linkedin.com/in/sandradiazquiroz}
}
\cvlistitem
{
    \textbf{Breno Colom}
    \newline{}
    Ingeniero de Software en Scrapinghub
    \newline{}
    \url{https://linkedin.com/in/brenocolom}
}
\end{document}
