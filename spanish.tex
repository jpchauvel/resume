%% start of file `Jean-Pierre.Chauvel.Résumé.tex`
%% Copyright 2014 Jean-Pierre Chauvel <jchauvel@gmail.com>

\documentclass[11pt,a4paper,english]{moderncv}
% moderncv theme
\moderncvtheme[grey]{casual}
% document language
\usepackage[english]{babel}
% character encoding
\usepackage[utf8]{inputenc}
% adjust page margins
\usepackage[scale=0.8]{geometry}
% in preamble:
\SetExpansion[stretch = 70, shrink = 70,] { encoding = {T2A} } { }
\DeclareMicrotypeSet{t2atext}{encoding=T2A}
\UseMicrotypeSet{t2atext}

%personal data
\firstname{Jean-Pierre Chauvel}
\familyname{}
\address{Lima, Perú}{}
%\email{jchauvel@gmail.com}
\email{jean.p.chauvel@gmail.com}
%\social[twitter]{hellhovnd}
\social[linkedin]{helhound}
\social[github]{hellhovnd}
\quote{Tengo más de 10 años de experiencia profesional desarrollando software de calida, he estado trabajando en equipos distribuidos (teletrabajo) desde el 2012 y me integro con facilidad en equipos de trabajo con metodologías ágiles.}

\begin{document}
\maketitle

\section{Experiencia}
\cventry{2014}{Desarrollador de software en iOS y Python}{PASTPRESENTFUTURE}{Telecommute, Lima--Perú}{}
{
\url{http://pastpresentfuture.com}
\newline{}
\begin{itemize}
    \item Doccupations
        \newline
        WIP (\url{http://doccupations.com})
        \newline
        Doccupations es una bolsa de trabajo pensada para dentistas. Usé Celery y RabbitMQ para coordinar emails asíncronos y búsquedas fulltext. Usé Haystack y Apache Solr para manejar las búsquedas fulltext de ofertas de empleo así como los matches automáticos de perfil de empleado con empleador.
\end{itemize}
}

\subsection{}

\cventry{2013--2014}{Desarrolador de Software en Python}{Bit Zeppelin}{Lima--Perú}{}
{
\url{http://bitzeppelin.com}
\newline{}
\begin{itemize}
    \item Sistema de Panificación de Recursos Empresariales (ERP)
        \newline{}
        Trabajé desarrollando el núcleo de la interfaz web de una aplicación desktop ERP basada en Tryton. Adicionalmente creé crawlers y spiders pensados para extraer datos de diferentes sitios web para la versión demo del ERP.
    \item Crédito Hipotecario
        \newline{}
        Crédito Hipotecario BCP es una web pensada para ofrecer servicios hipotecarios de proyectos del BCP y de teceros. Fude desarrollada enteramente utilizando TOrnado y MongoDB.
    \item Bolsa de trabajo UPC
        \newline{}
        Es una bolsa de trabajo web de la Universidad Peruana de Ciencias Aplicadas tanto para estudiantes y egresados.
\end{itemize}
}

\subsection{}

\cventry{2012--2013}{Desarrollador de software de iOS y Python}{Taller Technologies}{Lima--Perú}{}
{
\url{http://tallertechnologies.com}
\newline{}
\begin{itemize}
    \item Staff Allocation Tool (Herramienta de Asignación de Staff)
        \newline{}
        Staff Allocation Tool es una aplicación web hecha en Spring con Java pensada para administrar recursos humanos.
    \item Context Aware Computing (Servicio de Computación Sensible al Contexto)
        \newline{}
        Un proyecto aún en curso de Intel. Invertí mi tiempo más que nada como especialista y mentor de Python para el equipo de desarrollo.
    \item Kohl's Campus Kiosk (Quiosco del Campus para Kohl's)
        \newline{}
        Esta es una solución de software para que estudiantes planifiquen su prácticas preprofesionales en Kohl's. Estuve a cargo del frontend para iPad.
    \item TList (Ahora Chatpass.me)
        \newline{}
        TList es una app social para iOS pensada para el mercado de menores de edad. Tiene funciones similares a las apps LINE y Whatsapp del App Store de iOS. Fui el arquitecto de software y el desarrollador prinipal del equipo de desarrollo para el frontend.
\end{itemize}
}

\subsection{}

\cventry{2010--2012}{Desarrollador de software de iOS/Administrador de Proyectos/Socio}{Bit Zeppelin}{Lima--Perú}{}
{
\url{http://bitzeppelin.com}
\newline{}
\begin{itemize}
    \item App mitiempo.pe para Interbank. mitiempo.pe es una app que permite al usuario visualizar los eventos de entretenimiento así como administrar sus Favoritos.
        \newline{}
        Published \url{http://j.mp/mitiempo_pe}
        \newline{}
    \item App tootalk para Global Backbone. Esta es una app frontend hecha enteramente para iOS que permite al usuario comunicarse con la telefonía fija peruana desde cualquier punto geofráfico.
        \newline{}
        Published \url{http://j.mp/tootalk}
    \item App Lista de Compras para E. Wong
        \newline{}
        Not published
    \item App Teleticket para Interbank.
        \newline{}
        Published \url{http://j.mp/teleticket-ios}
    \item App de Lista de Compras para Plaza Vea. Esta es una app iOS, Android y Blackberry que al usuario visualizar los productos de Plaza Vea y administrar su propia lista de compras.
        \newline{}
        Not Published
\end{itemize}
}

\subsection{}

\cventry{2010}{Desarrollador de Software de Python y C++}{ICTEC}{Lima--Perú}{}
{
\url{http://itctec.biz}
\newline{}
\begin{itemize}
    \item Manejador de software especial para equipos militares para el Ejército Peruano.
        \newline{}
        Orion es una aplicación móvil de Control y Comando Conjunto utilizado por el Ejército Peruano. El manejador de software que desarrollé permitía al sistema comunicarse entre Radios VHF Tardiran CNR 9000 Israelíes enviar mapas tácticos y mensajes cortos.
    \item Submódulo de reventa para el portal web NeoAuto.
        \newline{}
        \url{http://neoauto.pe}
\end{itemize}
}

\subsection{}

\cventry{2009}{Desarrollador de Software en Python}{Grupo El Comercio}{Lima--Perú}{}
{
\url{http://grupoelcomercio.com.pe}
\newline{}
\begin{itemize}
    \item Mantenimiento de la aplicación web Aptitus.
        \newline{}
        \url{http://aptitus.pe}
    \item Mantenimiento de la aplicación web Neoauto.
        \newline{}
        \url{http://neoauto.pe}
\end{itemize}
}

\subsection{}

\cventry{2006--2009}{Desarrollador de Software de Python}{Aureal}{Lima--Perú}{}
{
\url{http://aureal.pe}
\newline{}
\begin{itemize}
    \item Aptitus: Una web de bolsa de trabajo para Grupo El Comercio.
    \newline{}
    \url{http://aptitus.pe}
    \item Neoauto: Una web de ventas de automotores para Grupo El Comercio.
    \newline{}
    \url{http://neoauto.pe}
    \item Relax: Una web de contenido para adultos para Grupo El Comercio.
    \newline{}
    Never published
    \item Tumax: Una plataforma de networking social para Grupo El Comercio.
    \newline{}
    Never published
    \item Sistema de Finanzas y Contabilidad para Americatel.
\end{itemize}
}

\subsection{}

\cventry{2004--2006}{Desarrollador de Componentes de Sofware COM en C++}{Applisys}{Lima--Perú}{}
{
\url{http://applisys.com.pe}
\newline{}
Mi función principal estuvo enfocada en desarrollar componentes de software para su ERP APPLISIG.
}

\subsection{}

\cventry{2003-2004}{Desarrollador de Softwware en PHP}{OyM System Group}{Asunción--Paraguay}{}
{
\url{http://oym.com.py}
\newline{}
Mi función principal estuvo enfocada en la migración de su framework ERP de FoxPro para DOS a Visual FoxPro 9.
}

\subsection{}

\cventry{2002}{Desarrollador de Software Junior}{Tecniaduanas}{Lima--Perú}{}
{
\url{http://tli.com.pe}
\newline{}
Mi función principal estuvo enfocada en la corrección de errores y mantenimiento de su ERP.
}

\subsection{}

\section{Educación}
\cventry{1999--2002}
    {Título en Ciencias de la Computación}
    {Instituto Superior Tecnológico de Ciencias de la Información}{Lima--Perú}
    {}{Ex-Instituto de Ciencias de la Información de la Universidad Nacional de Ingeniería}

\subsection{}

\cventry{1996--1998}
    {Diploma}
    {Colegio de la Inmaculada}{Lima--Perú}
    {}{}

\subsection{}
    
\section{Habilidades}

\subsection{Idiomas}
\cvlanguage{Inglés}{avanzado}{}
\cvlanguage{Español}{Nativo}{}

\subsection{}

\subsection{Procesos de Desarrollo}
\cvlanguage{Scrum}{Avanzado}{Todos mis proyectos en iOS utilizaron esta metodología}
\cvlanguage{Kanban}{Intermedio}{}
\cvlanguage{CMMI}{Básico}{Applisys ajustó sus procesos de desarrollo al nivel de capacidada 2}

\subsection{}

\subsection{Programación}
\cvlanguage{Objective-C}{Avanzado}{Para las plataformas iOS and OS X}
\cvlanguage{C++}{Intermedio}{Boost y Microsoft COM}
\cvlanguage{Python}{Avanzado}{Desarrollo web, scraping/crawling, mobile y desktop}

\subsection{}

\subsection{Frameworks de Test Unitario para Plataformas Objective-C/iOS/OS X}
\cvlanguage{OCUnit}{Avanzado}{Incluido en Xcode como SenTestingKit}
\cvlanguage{OCMock}{Avanzado}{\url{http://ocmock.org}}
\cvlanguage{OCHamcrest}{Avanzado}{\url{https://github.com/hamcrest/OCHamcrest}}
\cvlanguage{OCMockito}{Avanzado}{\url{https://github.com/jonreid/OCMockito}}
\cvlanguage{GHUnit}{Avanzado}{\url{https://github.com/gabriel/gh-unit}}
\cvlanguage{GTM}{Intermedio}{\url{http://code.google.com/p/google-toolbox-for-mac/}}
\cvlanguage{Kiwi}{Básico}{\url{https://github.com/allending/Kiwi}}
\cvlanguage{Cedar}{Básico}{\url{https://github.com/pivotal/cedar}}

\subsection{}

\subsection{Frameworks de Test Unitario para Python}
\cvlanguage{unittest}{Avanzado}{De la librería estándar de Python}
\cvlanguage{pytest}{Avanzado}{\url{http://pytest.org/}}
\cvlanguage{mockito}{Avanzado}{\url{https://code.google.com/p/mockito-python/}}

\subsection{}

\subsection{Test de Aceptación de Usuario para iOS}
\cvlanguage{UI Automation}{Intermedio}{Framework de Xcode, ejecutado con Instruments}
\cvlanguage{Frank}{Intermedio}{\url{http://www.testingwithfrank.com}}
\cvlanguage{iCuke}{Intermedio}{\url{https://github.com/unboxed/icuke}}
\cvlanguage{Zucchini}{Avanzado}{\url{http://www.zucchiniframework.org}}

\subsection{}

\subsection{Test de Aceptación de Usuario para Python}
\cvlanguage{doctest}{Avanzado}{Parte de la librería estándar de Python}
\cvlanguage{Lettuce}{Intermedio}{Un DSL similar a Cucumber y framework de test funcional \url{http://lettuce.it}}
\cvlanguage{Cucumber}{Intermedio}{Vía RubyPython \url{http://cukes.info}}
\cvlanguage{Selenium}{Avanzado}{Framework de automatización web \url{http://seleniumhq.org}}

\subsection{}

\subsection{Frameworks de Inyección de Dependencias de Objective-C para iOS/OS X}
\cvlanguage{Objection}{Intermedio}{\url{http://objection-framework.org}}
\cvlanguage{Typhoon}{Avanzado}{\url{http://www.typhoonframework.org}}

\subsection{}

\subsection{Distintos Frameworks de Objective-C para Entornos iOS/OS X}
\cvlanguage{NimbusKit}{Avanzado}{\url{http://nimbuskit.org}}
\cvlanguage{Three20}{Avanzado}{\url{http://three20.info}}
\cvlanguage{ASIHTTP Request}{Avanzado}{\url{http://allseeing-i.com/ASIHTTPRequest}}
\cvlanguage{AFNetworking}{Avanzado}{\url{http://afnetworking.com}}
\cvlanguage{cocos2d}{Avanzado}{\url{http://www.cocos2d-iphone.org}}

\subsection{}

\subsection{Distintos Frameworks de Python}
\cvlanguage{Django}{Avanzado}{\url{https://www.djangoproject.com}}
\cvlanguage{Tornado}{Avanzado}{\url{http://www.tornadoweb.org}}
\cvlanguage{Twisted}{Intermedio}{\url{https://twistedmatrix.com}}
\cvlanguage{Scrapy}{Avanzado}{\url{http://scrapy.org}}
\cvlanguage{PyQt}{Avanzado}{\url{https://qt-project.org}}
\cvlanguage{PyGTK}{Intermedio}{\url{http://www.pygtk.org}}
\cvlanguage{wxPython}{Avanzado}{\url{https://www.wxwidgets.org}}
\cvlanguage{Celery}{Avanzado}{\url{http://www.celeryproject.org}}
\cvlanguage{Haystack}{Avanzado}{\url{http://haystacksearch.org}}

\subsection{}

\subsection{AMPQ Message Brokers}
\cvlanguage{Apache Apollo}{Intermedio}{\url{https://activemq.apache.org/apollo/}}
\cvlanguage{RabbitMQ}{Avanzado}{\url{http://rabbitmq.com}}

\subsection{}

\subsection{Motores de Búsqueda Full-Text}
\cvlanguage{Apache Solr}{Avanzado}{\url{http://lucene.apache.org/solr/}}
\cvlanguage{Elasticsearch}{Avanzado}{\url{http://elasticsearch.org/}}
\cvlanguage{Xapian}{Avanzado}{\url{http://xapian.org/}}

\subsection{}

\subsection{Sistemas de Versionamiento}
\cvlanguage{Git}{Avanzado}{Configuración y uso}
\cvlanguage{Mercurial}{Avanzado}{Configuración y uso}
\cvlanguage{Subversion /CVS}{Avanzado}{Configuración y uso}
\cvlanguage{FishEye}{Básico}{(\url{http://www.atlassian.com/software/fisheye}) Solo uso}

\subsection{}

\subsection{Trackers de Trabajo y Registro de Horas}
\cvlanguage{Redmine}{Avanzado}{(\url{http://redmine.org}) Configuración y uso}
\cvlanguage{Trac}{Avanzado}{(\url{http://trac.edgewall.org}) Configuración y uso}
\cvlanguage{JIRA}{Avanzado}{(\url{http://www.atlassian.com/software/jira}) Configuración y uso}

\subsection{}

\subsection{Motores y Frameworks de Documentación}
\cvlanguage{Doxygen}{Avanzado}{}
\cvlanguage{Sphinx}{Avanzado}{\url{http://sphinx.pocoo.org}}
\cvlanguage{\LaTeX}{Avanzado}{Este Curriculum Vitae fue hecho en \LaTeX}

\subsection{}

\subsection{Sistemas Operativos}
\cvlanguage{GNU/Linux}{Sysadmin}{Solo en sistemas operativos basado en Debian}
\cvlanguage{BSD}{Sysadmin}{Cualquier BSD, incluyendo Mac OS X}
\cvlanguage{Windows}{Avanzado}{La familia de Windows NT 5.x e inferiores}

\subsection{}

\subsection{DBMS/Storage engines}
\cvlanguage{MySQL}{Avanzado}{}
\cvlanguage{PostgreSQL}{Intermedio}{}
\cvlanguage{SQL Server}{Básico}{2005 y anteriores}
\cvlanguage{Redis}{Intermedio}{\url{http://redis.io}}
\cvlanguage{MongoDB}{Avanzado}{\url{http://www.mongodb.org}}

\subsection{}

\subsection{Servicios de Integración Contínua}
\cvlanguage{Jenkins /Hudson}{Avanzado}{\url{http://jenkins-ci.org}}
\cvlanguage{Bamboo}{Intermedio}{\url{http://www.atlassian.com/software/bamboo}}
\cvlanguage{Travis}{Avanzado}{\url{https://travis-ci.org}}
\cvlanguage{CicleCI}{Avanzado}{\url{https://circleci.com}}
\cvlanguage{Fabric}{Avanzado}{\url{http://fabfile.org}}

\clearpage

\section{Referencias}
\cvlistitem
{
    \textbf{Bill Forbes}
    \newline{}
    CTO en Talent Trust
    \newline{}
    \url{https://linkedin.com/in/billforbes}
}
\cvlistitem
{
    \textbf{Michael Fromm}
    \newline{}
    Socio y CTO en PASTPRESENTFUTURE
    \newline{}
    \url{https://linkedin.com/pub/michael-fromm/0/48a/450}
}
\cvlistitem
{
    \textbf{Mandeep Dhillon}
    \newline{}
    Promotor de Tecnologías para Niños
    \newline{}
    \url{https://linkedin.com/in/mandeepdhillon}
}
\cvlistitem
{
    \textbf{Miguel Zabludovsky}
    \newline{}
    Presidente y CEO en Slate NYC
    \newline{}
    \url{https://www.linkedin.com/in/miguelzabludovsky}
}
\cvlistitem
{
    \textbf{Breno Colom}
    \newline{}
    Ingeniero de Software en Scrapinghub
    \newline{}
    \url{https://linkedin.com/in/brenocolom}
}
\end{document}
